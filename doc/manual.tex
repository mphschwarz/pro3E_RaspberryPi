\documentclass[a4paper]{article}

\usepackage[utf8]{inputenc}
\usepackage[T1]{fontenc}
\usepackage{lmodern}
\usepackage[nswissgerman]{babel}
\usepackage{hyperref}
\usepackage[top=1.5cm, bottom=1.5cm, right=1.5cm, left=1.5cm]{geometry}

\title{Betriebsanleitung}
\author{Michael Schwarz}

\begin{document}

\maketitle

\section{Start-Up}
\begin{enumerate}
\item Den Verbraucher in die Buchse des Messgeräts einstecken und das Messgerät in eine Stromsteckdose einstecken.
\item Mit dem Schalter den gewünschten Strommessbereich auswählen: \\
	\begin{tabular}{c l}
		(I) & für Ströme bis 1 A \\
		(II) & für Ströme über 1 A.
	\end{tabular}
\item Leuchtet die StatusLED ist das Messgerät in Betrieb.
\end{enumerate}

\section{Shutdown}
\begin{enumerate}
	\item Den Shutdown-Taster für 5 s drücken, Taster loslassen und 10 s warten.
	\item Verbraucher und Messgerät trennen.
\end{enumerate}

\section{Daten auslesen}
\begin{enumerate}
	\item Computer oder Mobiltelefon mit W-LAN-hotspot des Messgeräts verbinden. \\
		\begin{tabular}{l l}
			SSID: & WiPi \\
			Passwort: & blackberry
		\end{tabular}
	\item Per Internetbrowser (Google Chrome oder Mozilla Firefox) auf die IP-Adresse des W-Lan-hotspots navigieren: \\
		\url{192.168.42.1}
	\item Für die neusten Daten muss der Browser einen hard-refresh machen: Ctrl+f5
	\item Die laufende Messung wird oben mit zwei Plots beschrieben. 
		Ältere Messdaten sind unter den Plots verlinkt.
		Die von den äleteren Messdaten sind Textdateien sowie Plots verfügbar.
\end{enumerate}
\end{document}
